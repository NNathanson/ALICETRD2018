%% LyX 2.3.0 created this file.  For more info, see http://www.lyx.org/.
%% Do not edit unless you really know what you are doing.
\documentclass[english]{article}
\usepackage[T1]{fontenc}
\usepackage[latin9]{inputenc}
\usepackage{babel}
\begin{document}

\section{Suggestions for future investigations}

\subsection{More sophisticated ray reconstruction algorithm}

There are various assumptions made in the ray reconstruction algorithm
which may be considered questionable. First of all, the influence
of an incident electron on the pads above it is not a time-localized
phenomenon, the electron will tend to influence the pads for a couple
time bins before it actually reaches the anode wire. Additionally,
the way in which an electron affects the pads not directly above it
is also a non-trivial one and because of this, the linear fits with
linear ADC weighting may be called into question. Another key relationship
is between the deposition depth of an electron and the corrosponding
time bins in which it leaves its influence. This may require further
investigation, especially in regions with non-constant electric fields,
such as the anode zone. Finally, efforts ought to be made in the future
to filter out unreliable fits due to multiple muons being caught within
a single event in different locations.

\subsection{Energy deposition}

Another interesting relationship which may be worth investigating
is the link between the ADC values and the associated energy deposition
in the gas due to a muon. This is likely to be a complicated relationship
involving the Bethe-Bloch equation and various properties of the amplification
zone but this ought to be doable and would provide an interesting
method of verifying whether muons are the true cause of the tracks
in the detector.

\subsection{Investigating gas purity as an explination for pulse height spectrum
plateau inclination.}

The pulse height spectra in sections ??? demonstrate a slanted plateau
corrosponding to the drift region. This is hypothesised to be due
to electron re-absorbtion by various impurities in the gas such as
Oxygen. Perhaps a method of measureing the purity of the gas could
be developed in the future and used to verify or reject this prediction.

\subsection{Multiple scintillation co-incidence triggering}

In the current configuration, the scintillation detector is placed
underneath the TRD chamber. This leaves the configuration vulnerable
to triggers from muons which pass through the scintillation detector
but not through the TRD chamber (at a steep angle). Therefore placing
a second detector above the TRD chamber and only triggering on co-incidence
between the top and bottom detectors will help to reduce false triggers.
Having multiple scintillators will also tend to reduce false triggers
due to noise and other non-muon scintillation events.

\subsection{Muon shower origin triangulation}

A future project may look into using multiple scintillators and the
relative delay between signals from a muon shower to calculate the
origin of the shower in the upper atmosphere. This may also be compared
with the angle of the tracklets through the TRD chamber ton see if
the points coincide.

\subsection{TRD efficiency}

There is currently no way of calculating what the probability of detecting
an incidant muon is provided it passes through the detector and/or
scintillator this may be worth investigating in the future.

\subsection{Measure local cosmic muon flux}

Another interesting project may be to measure the total local flux
density of cosmic muons, possibly as a function of solid angle when
coupled with suggestion 5, in Cape Town. This may be performed with
multiple scintillation detectors and compared to other common literature
values.
\end{document}
